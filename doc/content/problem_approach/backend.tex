\section{Backend}
\label{sec:backend}

The backend of the Sapphire Wallet is a set of microservices that delegate and simplify the interaction between the user and the Blockchain. Moreover, the backend is the \textbf{Relayer} of the Sapphire Wallet ecosystem. The general process includes the \hyperref[sec:mobile_application]{Mobile Application} sending a signed transaction to the Backend, which will execute (and pay) it on behalf of the user.

As shown in the figure \ref{fig:problem_approach/backend_infrastructure}, the Backend is composed of the following components:
\begin{itemize}
    \item \textit{API Gateway}: Entry point of the Backend. It is responsible for the routing of the requests to the correct microservice.
    \item \textit{Wallet Factory}: Creates the user wallet on the \textit{Base Chain}.
    \item \textit{Sapphire Relayer}: Executes the transactions on behalf of the user.
    \item \textit{Sapphire Portfolio}: Retrieves the user wallet information on all the chains.
\end{itemize}

\borderfigure{problem_approach/backend_infrastructure}{Sapphire Wallet Backend Infrastructure}{1}

The backend is developed in Typescript and uses the \textit{NestJS}\footnote{https://nestjs.com/} framework. The communication between the microservices is done via \textit{TCP} protocol to ensure a better performance. 

\subsection{API Gateway}
\label{subsec:api_gateway}

The API Gateway is the entry point of the Backend. It is responsible for the routing of the requests to the correct microservice. The communication is done using NestJs Event-Based Pattern. 

The API Gateway provides \textit{Swagger}\footnote{https://swagger.io/} OpenAPI documentation.

\subsection{Wallet Factory}
\label{subsec:wallet_factory}

The Wallet Factory is the microservice that creates the user wallet on the \textit{Base Chain}. The user wallet is created using the \textit{WalletFactory} smart contract.  

The Microservice will wait until the transaction is fully confirmed on the Blockchain before returning the wallet address to the user.

\subsection{Sapphire Relayer}
\label{subsec:sapphire_relayer}

The Sapphire Relayer has two main tasks:

\begin{itemize}
    \item \textit{Execute} the transactions on behalf of the user. The transactions are sent by the \hyperref[sec:mobile_application]{Mobile Application} signed by the user. The Relayer will execute the transaction and pay the gas fee.
    \item \textit{Authorise} new wallets address. Each Sapphire Wallet has a list of authorised wallets with which it can interact.
\end{itemize}

The Microservice will wait until the transaction is fully confirmed on the Blockchain before returning a confirmation or an error to the user.

\subsection{Sapphire Portfolio}
\label{subsec:sapphire_portfolio}

In a Multi-Chains context, the retrieval of the wallet information is a complex task. Especially because the ownership of NFT and tokens are stored in different smart contracts.

The Sapphire Portfolio is the microservice that retrieves the wallet info on all the chains. The microservice will firstly fetch the \textit{AccountContract} address of each \textit{Dest Chain} used by the user. Then, it will retrieve the NFTs and the tokens owned by the user on each chain. Finally, it will return the wallet info to the user.