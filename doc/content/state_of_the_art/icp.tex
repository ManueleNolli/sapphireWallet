\section{IC: Internet Computer}
\label{sec:icp}

The Internet Computer (IC)\footnote{https://internetcomputer.org/} is a blockchain protocol developed by the DFINITY Foundation\footnote{https://dfinity.org/}, a Switzerland non-profit organization.  

IC's vision is that most of the world's software will be replaced by smart contracts. To realize that vision, IC is designed to make smart contracts as powerful as traditional software. \cite{icp-vision}

\subsection{Architecture}

The Internet Computer (IC) realizes the vision of a \textit{World Computer}, an open and secure blockchain-based network that can host programs and data in the form of smart contracts, perform computations on smart contracts in a secure and trustworthy way, and scale infinitely.\cite{icp-how-it-works}

Smart contracts on the Internet Computer are called \textbf{Canister} smart contracts, each consisting of a bundle of WebAssembly (Wasm) bytecode and storage. Each canister has its own, isolated, data storage that is only changed when the canister executes code. \cite{icp-how-it-works}

The IC is designed to be highly scalable and efficient in terms of hosting and executing canister smart contracts. The top-level building blocks of the IC are subnetworks, or \textbf{subnets}, the IC as a whole consists of many subnets, where each subnet is its own blockchain that operates concurrently with and independently of the other subnets (but can communicate asynchronously with other subnets). Each subnet hosts canister smart contracts, up to a total of hundreds of gigabytes of replicated storage. A subnet consists of node machines, or nodes. Each node replicates all the canisters, state, and computation of the subnet using blockchain technology. This makes a subnet a blockchain-based replicated state machine, that is, a virtual machine that holds state in a secure, fault-tolerant, and non-tamperable manner. The computations of the canisters hosted on a subnet will proceed correctly and without stopping, even if some nodes on the subnet are faulty (either because they crash, or even worse, are hacked by a malicious party). New subnets can be created from nodes added to the IC to scale out the protocol, analogous to how traditional infrastructures such as public clouds scale out by adding machines. \cite{icp-architecture}

\textcolor{red}{Add figure}

\subsubsection{IC Protocol (ICP)}

\textcolor{red}{Go deeper or remove?}

The Internet Computer is created by the Internet Computer Protocol (ICP). The ICP is a 4-layer protocol that is running on the nodes of each subnet. The 4 layers are: \cite{icp-how-it-works}

\begin{itemize}
    \item \textit{Peer-to-peer}: Responsible for the secure and reliable communication between the nodes of a subnet. Using P2P, a node can broadcast a network message to all the nodes in the subnet.
    \item \textit{Consensus}: Allows the nodes to agree on the messages to be processed, as well as their ordering. Consensus is the component of the core IC protocol that drives the subnets of the IC.
    \item \textit{Message routing}: Receives a block of messages to be processed from consensus and places the messages into the input queues of the target canisters.
    \item \textit{Execution}: execute canister smart contract code.
\end{itemize}


\subsection{Internet Identity: Account Abstraction}

Internet Identity is the ICP's native form of identity. IC uses \textit{passkeys}, a unique public/private key pair stored in the secure hardware chip, to authenticate users. They offer a convenient and more secure alternative to passwords, enabling a password-free login experience for websites and applications. \cite{icp-identity}

Passkeys are based on the WebAuthentication (WebAuthn) cryptographic standard. When creating an account, the operating system generates a unique and app- or website-specific cryptographic key pair \textit{on the device}. \cite{icp-passkey}

To interact with a canister, users need to authenticate their actions using a cryptographic key pair (passkey). The network verifies the digital signature using the associated public key, ensuring that the action was indeed signed by the authorized user. If the verification is successful, the action is executed by the canister on the Internet Computer. \cite{icp-passkey}

Internet Identity offers several benefits over other blockchains account systems, some \hyperref[sec:account_abstraction]{\textit{Account Abstraction}} features are natively supported: \cite{icp-identity-technical}
\begin{itemize}
    \item \textit{Improved User Experience}: Authentication via cryptographic key on the device without the need of passwords.
    \item \textit{Security}: Users can add many devices to the same identity \textit{Anchor}\footnote{An anchor is a number attached to a user's identity. This number is auto-generated by the system and there is only one per identity.}. A common example is that users may add their smartphone and desktop computer to an identity anchor. 
    \item \textit{Recovery Mechanism}: There are two recovery mechanisms for recovery the \textit{Anchor}:
        \begin{itemize}
            \item \textit{Seed Phrase}: A user can select this option to generate a cryptographically-secure seed phrase that they can use to recover an Identity Anchor. 
            \item \textit{Security Key}: A dedicated security key can be used to recover an Identity Anchor in the event that a user loses access to their authorized devices. 
        \end{itemize}
\end{itemize}

\subsection{Chain-key Cryptography: Interoperability}

\textit{Chain-key cryptography} enables subnets of the Internet Computer to jointly hold cryptographic keys, in a way that no small subset of potentially misbehaving nodes on the subnet can perform useful operations with the key, but the majority of honest nodes together can. \cite{icp-chain-key}

One of the major benefits of chain-key cryptography is that it enables canister to \textit{natively create signed transactions on other blockchains} like Ethereum or Bitcoin \textit{without a bridge}. \cite{icp-cross-chain-interoperability} With the combination of HTTPS outcalls feature and chain-key cryptography, the Internet Computer canisters can interact with external services and blockchains. \cite{icp-https-outcalls}

For instance, ICP is integrated with the Bitcoin network using chain-key ECDSA signatures and a protocol-level integration, allowing for a canister to create a Bitcoin address, then send or receive bitcoin directly as if they were a regular Bitcoin user. ICP will also use chain-key ECDSA to facilitate an upcoming integration with Ethereum that will allow Ethereum smart contracts and digital assets like ERC-20 tokens to be used in ICP canisters. \cite{icp-cross-chain-interoperability}

