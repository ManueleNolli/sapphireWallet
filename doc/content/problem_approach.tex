\chapter{Problem Approach: Sapphire Wallet}
\label{chap:problem_approach}

In this chapter, The Sapphire Wallet ecosystem is presented. The Sapphire Wallet is a Multi-Chain Account Abstraction, based on an improved version of the \hyperref[subsec:argent]{Argent Wallet}.

Sapphire Wallet is a Proof of Concept of the future of Blockchain technology. The main goal of Sapphire Wallet is to give an idea of the potential of the Blockchain technology in combination with Account Abstraction Layer and Multi-Chain Bridge approach. To achieve this goal, the Sapphire Wallet includes a set of features already present in the Argent Wallet, such as guardian, wallet creation and wallet recovery. Moreover, the Sapphire Wallet ecosystem includes a Multi-Chain Bridge, which allows the user to interact with different Blockchains. 

It is important to notice that the Sapphire Wallet is developed on the current version of EVM\footnote{Dencun version}. There are several promising \hyperref[subsec:eips]{EIPs} that could be implemented in the future, such as the \hyperref[subsubsec:eip-3074]{EIP-3074} and the \hyperref[subsubsec:erc-7702]{EIP-7702}, but they need an hard fork of the Ethereum network. Whatever the final Account abstraction implementation is, the Sapphire Wallet provides a realistic Proof of Concept of how the users will interact with the Blockchain in the future.

In the following sections, the architecture and the components of the Sapphire Wallet ecosystem are presented. Each section is described in isolation, but at the end of the chapter, the operational flow and the use cases are presented to give a complete overview of the Sapphire Wallet functionalities.

\section{Architecture}
\label{sec:architecture}

The Sapphire Wallet ecosystem is composed of four components:

\begin{itemize}
    \item \textit{Blockchain}: Two or more Blockchains networks of which one is called \textit{Base Chain} and the others are called \textit{Dest Chain(s)}. The Acccount Abstraction Layer is implemented on the \textit{Base Chain}.
    \item \textit{Bridge}: \textit{Lock \& Mint} bridge that allows the user to transfer assets and \textit{transactions} between the \textit{Base Chain} and the \textit{Dest Chain(s)}.
    \item \textit{Backend}: Infrastructure that simplifies the interaction between the Blockchains and the Mobile Application. Moreover, the Backend is the \textit{Relayer} that sends the transactions to the Blockchain networks.
    \item \textit{Mobile Application}: User interface that allows the user to interact with the Sapphire Wallet ecosystem.
\end{itemize}

A simplified architecture of the Sapphire Wallet Infrastructure is shown in Figure \ref{fig:problem_approach/infrastructure}. The specific components will be described in the following sections.

\borderfigure{problem_approach/infrastructure}{Simplified Sapphire Infrastructure}{1}
\section{Blockchain}
\label{sec:blockchain}

Various smart contracts are implemented  and deployed on the blockchain to manage the Sapphire Wallet. There is a distinction between the Base Chain and the Dest Chain(s):
\begin{itemize}
    \item \textit{Base Chain}: smart contracts to manage Account Creation, Wallet Recovery, Transaction execution, etc.
    \item \textit{Dest Chain(s)}: smart contracts to abstract the \textit{Base Chain} accounts and securely implement the bridge between the \textit{Base Chain} and the \textit{Dest Chain(s)}. 
\end{itemize}

\subsection{Base Chain}
\label{subsec:base_chain}

The Base Chain is the main blockchain where the Sapphire Wallet is deployed. The contracts are an extension of the \hyperref[subsec:argent]{Argent Wallet} contracts. The simplified architecture of the Base Chain is shown in the attachment \addattachment{SapphireWalletBlockchainSmartContracts.png}{SapphireWalletBlockchainSmartContracts}.

The main features of the Base Chain are:
\begin{itemize}
    \item \textit{Account Creation}: via \textit{WalletFactory} it is possible to deploy a new \textit{Proxy} contract per user. When the proxy is called, it delegates the call to the \textit{BaseWallet} contract. It is important to note that the \textit{Proxy} will maintain its own state, but it will use the implementation of the \textit{BaseWallet} contract. This pattern is used to reduce the gas of deployment for each user.
    \item \textit{Security} the combination of the \textit{SecurityManager} and \textit{GuardianManager} contracts allows the user to set up security features such as guardians, locks, whitelists, etc.
    \item \textit{Relayed Transactions}: the \textit{RelayerManager} and \textit{TransactionManager} contracts allow the user to execute one or batch transactions. The transactions can be payed by a relayer, so the user does not need to have the native token of the blockchain. 
    \item \textit{Interoperability}: the \textit{InteroperabilityManager} contract allows the user to emit, after a security check, events. The events are then relayed to the \textit{Dest Chain(s)} by the bridge.
\end{itemize}

\textcolor{red}{say that all the process are well described in the section usecases}


For instance, it is interesting to analyse the process of a relayed ETH transfer transaction. The process is the following:
\begin{enumerate}
    \item The Relayer will call the \textit{RelayerManager} contract with the transaction data (including the user transaction and the related signature).
    \item The \textit{RelayerManager} will call the \textit{TransactionManager} contract to execute the transaction on behalf of the user.
    \item The \textit{TransactionManager} will invoke the \textit{Proxy} contract which in turn will use the implementation of the \textit{BaseWallet} contract but maintaining the context of the proxy contract.
\end{enumerate}

% talk about interoperability manager

An important enhancement to the Sapphire Portfolio with regard to argent contracts is the \textit{InteroperabilityManager} contract. This contract allows the user to emit events that are then relayed to the \textit{Dest Chain(s)} by the bridge. The events are emitted after a security check. The \textit{InteroperabilityManager} contract is a key component to enable the communication between the \textit{Base Chain} and the \textit{Dest Chain(s)}.

\subsection{Dest Chain(s)}
\label{subsec:dest_chain(s)} 

\section{Bridge}
\label{sec:bridge}

The bridge is a key component of the Sapphire Wallet ecosystem. It allows the user to interact with different Blockchains, as visible by the figure \ref{fig:problem_approach/bridge_architecture}. The bridge main purpose is to listen to the events emitted by the \textit{InteroperabilityManager} contract on the \textit{Base Chain} and relay them to the desired \textit{Dest Chain} \textit{SapphireWrappedAccounts} contracts. 

\borderfigure{problem_approach/bridge_architecture}{Bridge Architecture}{1}

More in detail, the bridge is a \textit{trusted Bridge} written in Typescript that uses the \textit{Lock \& Mint} asset transfer model. 

The Sapphire Bridge does not transfer only assets as the native tokens or NFTs, but also \textbf{general transactions}. Indeed, with the \hyperref[sec:blockchain]{Sapphire Account Abstraction infrastructure}, the \hyperref[sec:mobile_application]{Mobile Application} and the Bridge, the user has the impression that he is interacting with a single Blockchain, while in reality, he is interacting with multiple Blockchains. 
\section{Backend}
\label{sec:backend}

Microservices infrastructure.

\subsection{API Gateway}
\label{subsec:api_gateway}

\subsection{Wallet Factory}
\label{subsec:wallet_factory}

WalletCreation.draw.io

\subsection{Sapphire Relayer}
\label{subsec:sapphire_relayer}

\subsubsection{Base Chain Relayer}
\label{subsubsec:base_chain_relayer}

TransactionExecution.draw.io

\subsubsection{Dest Chain Relayer}
\label{subsubsec:dest_chain_relayer}

All bridgeCall folder

\subsection{Sapphire Portfolio}
\label{subsec:sapphire_portfolio}

Problem: how to retrieve the wallet info on all the chains?

portfolio folder

\section{Mobile Application: Sapphire Wallet}
\label{sec:mobile_application}

The Sapphire Wallet Mobile App is the main component of the Sapphire ecosystem. It is the interface that enable the users to interact seamlessly with the Blockchain. Indeed, it enables the users to manage their assets and perform transactions on multiple Blockchains without the need to know anything about the Blockchain technology.

The Mobile App maintains securely saved an External Owned Account keys, which are used to sign transactions that are then sent to the \hyperref[sec:backend]{Backend} to be forwarded to the \textit{Base Chain}. Each time a user wants to perform an action, the Mobile App will ask a biometric authentication to sign the transaction. The EOA keys are important because the grant the user the ownership of the \textit{Proxy} contract well described in the \hyperref[subsec:base_chain]{\textit{Base Chain}} section. But, in the case of loss of the EOA keys, the user can recover its wallet by using the \hyperref[subsec:wallet_recovery]{Wallet Recovery} process.

Sapphire Wallet is composed of three main pages:
\begin{itemize}
    \item \textit{Home Page}: Dynamically rendered based on the user's assets. It displays the user's balance for each chain, and provides buttons to perform transactions as \textit{Send Crypto / NFTs}.
    \item \textit{NFTs Page}: Displays the user's NFTs by highlighting which blockchain they are stored on.
    \item \textit{Settings Page}: Allows the user to manage their preferences, settings and guardians. 
\end{itemize}

In addition to these pages, the Sapphire Wallet also provides the functionality to create a new wallet or recover an existing one at the first launch.

The framework used to build the Sapphire Wallet is \textit{Expo}\footnote{https://expo.dev/}, which allows for the development of cross-platform applications. In fact, the Sapphire Wallet is available for both Android and iOS devices.

\subsection{Wallet Creation}
\label{subsec:wallet_creation}

\subsection{Wallet Recovery}
\label{subsec:wallet_recovery}

\subsection{Home Page}
\label{subsec:wallet_home_page}

Write all the operations, balance of chains, button.

\subsection{NFTs Page}
\label{subsec:nfts_page}

\subsection{Settings Page}
\label{subsec:settings_page}

guardian


\section{Operational Flow and Use Cases}

\subsection{Base Chain}

\subsubsection{Wallet Creation}

% WalletCreation
\borderfigure{problem_approach/flows/walletCreation}{Sequence Diagram: Wallet Creation}{0.8}

\subsubsection{Wallet Recovery}

% WalletRecovery
\borderfigure{problem_approach/flows/walletRecovery}{Sequence Diagram: Wallet Recovery}{0.8}

% improvedWalletRecovery
\borderfigure{problem_approach/flows/improvedWalletRecovery}{Sequence Diagram: Improved Wallet Recovery}{0.8}

\subsubsection{General Flow}

% TransactionsExecution
\borderfigure{problem_approach/flows/transactionsExecution}{Sequence Diagram: General Transaction execution}{0.8}


% Explain that this works for crypto transfer and for general transactions (e.g. smart contracts calls).


\subsection{Portfolio}

\subsubsection{Balance}

% portfolioBalance
\borderfigure{problem_approach/flows/portfolioBalance}{Sequence Diagram: Retrieve Native crypto balance in a Multi-Chain environment}{0.8}

\subsubsection{NFTs}

% portfolioNFTs
\borderfigure{problem_approach/flows/portfolioNFTs}{Sequence Diagram: Retrieve owned NFTs in a Multi-Chain environment}{0.8}

\subsection{Multi-Chain}

\subsubsection{Native Crypto Transfer}

% bridgeETHBaseChainToDestChain
\borderfigure{problem_approach/flows/bridgeETHBaseChainToDestChain}{Sequence Diagram: Bridge Native Crypto from \textit{Base Chain} to \textit{Dest Chain}}{1.}

% bridgeETHDestChainToBaseChain
\borderfigure{problem_approach/flows/bridgeETHDestChainToBaseChain}{Sequence Diagram: Bridge Native Crypto from \textit{Dest Chain} to \textit{Base Chain}}{1.}

% bridgeETHRollback
\borderfigure{problem_approach/flows/bridgeETHRollback}{Activity Diagram: Rollback on bridge Native Crypto failure}{1.}

\subsubsection{NFTs Transfer}

% bridgeNFTBaseChainToDestChain
\borderfigure{problem_approach/flows/bridgeNFTBaseChainToDestChain}{Sequence Diagram: Bridge NFT from \textit{Base Chain} to \textit{Dest Chain}}{1.}

% bridgeNFTDestChainToBaseChain
\borderfigure{problem_approach/flows/bridgeNFTDestChainToBaseChain}{Sequence Diagram: Bridge NFT from \textit{Dest Chain} to \textit{Base Chain}}{1.}

% bridgeNFTRollback
\borderfigure{problem_approach/flows/bridgeNFTRollback}{Activity Diagram: Rollback on bridge NFTs failure}{1.}


\subsubsection{Generic transactions}

% bridgeGenericDestChainToDestChain
\borderfigure{problem_approach/flows/bridgeGenericDestChainToDestChain}{Sequence Diagram: Bridge Generic TX}{1.}

% bridgeGenericRollback
\borderfigure{problem_approach/flows/bridgeGenericTXRollback}{Activity Diagram: Rollback on bridge TX failure}{1.}

