\chapter{Problem Approach: Sapphire Wallet}
\label{chap:problem_approach}

In this chapter, The Sapphire Wallet ecosystem is presented. The Sapphire Wallet is a Multi-Chain Account Abstraction wallet, based on an improved version of the \hyperref[subsec:argent]{Argent Wallet}.

The main goal of Sapphire Wallet is to give an idea of the potential of the Blockchain technology in combination with Account Abstraction Layer and Multi-Chain Bridge approach. To achieve this goal, the Sapphire Wallet includes a set of features already present in the Argent Wallet, such as guardian, wallet creation and wallet recovery. In addition, the Sapphire Wallet ecosystem includes a Multi-Chain Bridge, which allows the user to interact with different Blockchains. 

It is important to notice that the Sapphire Wallet is developed on the current version of EVM\footnote{Dencun version}. There are several promising \hyperref[subsec:eips]{EIPs} that could be implemented in the future, such as the \hyperref[subsubsec:eip-3074]{EIP-3074} and the \hyperref[subsubsec:erc-7702]{EIP-7702}, but they need an hard fork of the Ethereum network. Whatever the final Account abstraction implementation is, the Sapphire Wallet provides a realistic Proof of Concept of how the users will interact with the Blockchain in the future.

In the following sections, the architecture and the components of the Sapphire Wallet ecosystem are presented. Each section is described in isolation, but at the end of the chapter, the operational flow and the use cases are presented to give a complete overview of the Sapphire Wallet functionalities.

\section{Architecture}
\label{sec:architecture}

The Sapphire Wallet ecosystem is composed of four components:

\begin{itemize}
    \item \textit{Blockchain}: Two or more Blockchains networks of which one is called \textit{Base Chain} and the others are called \textit{Dest Chain(s)}. The Acccount Abstraction Layer is implemented on the \textit{Base Chain}.
    \item \textit{Bridge}: \textit{Lock \& Mint} bridge that allows the user to transfer assets and \textit{transactions} between the \textit{Base Chain} and the \textit{Dest Chain(s)}.
    \item \textit{Backend}: Infrastructure that simplifies the interaction between the Blockchains and the Mobile Application. Moreover, the Backend is the \textit{Relayer} that sends the transactions to the Blockchain networks.
    \item \textit{Mobile Application}: User interface that allows the user to interact with the Sapphire Wallet ecosystem.
\end{itemize}

A simplified architecture of the Sapphire Wallet Infrastructure is shown in Figure \ref{fig:problem_approach/infrastructure}. The specific components will be described in the following sections.

\borderfigure{problem_approach/infrastructure}{Simplified Sapphire Infrastructure}{1}
\section{Blockchain}
\label{sec:blockchain}

Various smart contracts are implemented  and deployed on the blockchain to manage the Sapphire Wallet. There is a distinction between the Base Chain and the Dest Chain(s):
\begin{itemize}
    \item \textit{Base Chain}: smart contracts to manage Account Creation, Wallet Recovery, Transaction execution, etc.
    \item \textit{Dest Chain(s)}: smart contracts to abstract the \textit{Base Chain} accounts and securely implement the bridge between the \textit{Base Chain} and the \textit{Dest Chain(s)}. 
\end{itemize}

\subsection{Base Chain}
\label{subsec:base_chain}

The Base Chain is the main blockchain where the Sapphire Wallet is deployed. The contracts are an extension of the \hyperref[subsec:argent]{Argent Wallet} contracts. 

\subsection{Dest Chain(s)}
\label{subsec:dest_chain(s)} 

\section{Bridge}
\label{sec:bridge}

\section{Backend}
\label{sec:backend}

The backend of the Sapphire Wallet is a set of microservices that delegate and simplify the interaction between the user and the blockchain. Moreover, the backend is the \textbf{Relayer} of the Sapphire Wallet. Indeed, the \hyperref[sec:mobile_application]{Mobile Application} will send the signed transactions to the backend, which will execute (and pay) them on behalf of the user.

As shown in the figure \ref{fig:problem_approach/backend_infrastructure}, the backend is composed of the following components:
\begin{itemize}
    \item \textit{API Gateway}: the entry point of the backend. It is responsible for the routing of the requests to the correct microservice.
    \item \textit{Wallet Factory}: the microservice that creates the user wallet on the \textit{Base Chain}.
    \item \textit{Sapphire Relayer}: the microservice that executes the transactions on behalf of the user.
    \item \textit{Sapphire Portfolio}: the microservice that retrieves the user wallet information on all the chains.
\end{itemize}

\customfigure{problem_approach/backend_infrastructure}{Sapphire Wallet Backend Infrastructure}{1}

The backend is developed in Typescript and uses the \textit{NestJS}\footnote{https://nestjs.com/} framework. The communication between the microservices is done via \textit{TCP} protocol to ensure a better performance. 

\subsection{API Gateway}
\label{subsec:api_gateway}

The API Gateway is the entry point of the backend. It is responsible for the routing of the requests to the correct microservice. The communication is done using NestJs Event-Based Pattern. 

\subsection{Wallet Factory}
\label{subsec:wallet_factory}

The Wallet Factory is the microservice that creates the user wallet on the \textit{Base Chain}. The user wallet is created using the \textit{WalletFactory} smart contract.  

The Microservice will wait until the transaction is fully confirmed on the blockchain before returning the wallet address to the user.

\subsection{Sapphire Relayer}
\label{subsec:sapphire_relayer}

The Sapphire Relayer has two main tasks:

\begin{itemize}
    \item \textit{Execute} the transactions on behalf of the user. The transactions are sent by the \hyperref[sec:mobile_application]{Mobile Application} and are signed by the user. The Relayer will execute the transaction and pay the gas fee.
    \item \textit{Authorise} new wallets address. Each Sapphire Wallet has a list of authorised wallets with which it can interact.
\end{itemize}

The Microservice will wait until the transaction is fully confirmed on the blockchain before returning a confirmation or an error to the user.

\subsection{Sapphire Portfolio}
\label{subsec:sapphire_portfolio}

In a Multi-Chains context, the retrieval of the wallet information is a complex task. Especially because the ownership of NFT and tokens are stored in different smart contracts.

The Sapphire Portfolio is the microservice that retrieves the wallet info on all the chains. The microservice will firstly retrieve the \textit{AccountContract} address of each \textit{Dest Chain} used by the user. Then, it will retrieve the NFTs and the tokens owned by the user on each chain. Finally, it will return the wallet info to the user.
\section{Mobile Application: Sapphire Wallet}
\label{sec:mobile_application}

\subsection{Wallet Creation}
\label{subsec:wallet_creation}

\subsection{Wallet Recovery}
\label{subsec:wallet_recovery}

\subsection{Home Page}
\label{subsec:wallet_home_page}

Write all the operations, balance of chains, button.

\subsection{NFTs Page}
\label{subsec:nfts_page}

\subsection{Settings Page}
\label{subsec:settings_page}

guardian


\section{Operational Flow and Use Cases}

\subsection{Base Chain}

\subsubsection{Wallet Creation}

% WalletCreation
\customfigure{problem_approach/flows/walletCreation}{Sequence Diagram: Wallet Creation}{0.8}

\subsubsection{Wallet Recovery}

% WalletRecovery
\customfigure{problem_approach/flows/walletRecovery}{Sequence Diagram: Wallet Recovery}{0.8}

% improvedWalletRecovery
\customfigure{problem_approach/flows/improvedWalletRecovery}{Sequence Diagram: Improved Wallet Recovery}{0.8}

\subsubsection{General Flow}

% TransactionsExecution
\customfigure{problem_approach/flows/transactionsExecution}{Sequence Diagram: General Transaction execution}{0.8}


% Explain that this works for crypto transfer and for general transactions (e.g. smart contracts calls).


\subsection{Portfolio}

\subsubsection{Balance}

% portfolioBalance
\customfigure{problem_approach/flows/portfolioBalance}{Sequence Diagram: Retrieve Native crypto balance in a Multi-Chain environment}{0.8}

\subsubsection{NFTs}

% portfolioNFTs
\customfigure{problem_approach/flows/portfolioNFTs}{Sequence Diagram: Retrieve owned NFTs in a Multi-Chain environment}{0.8}

\subsection{Multi-Chain}

\subsubsection{Native Crypto Transfer}

% bridgeETHBaseChainToDestChain
\customfigure{problem_approach/flows/bridgeETHBaseChainToDestChain}{Sequence Diagram: Bridge Native Crypto from \textit{Base Chain} to \textit{Dest Chain}}{1.}

% bridgeETHDestChainToBaseChain
\customfigure{problem_approach/flows/bridgeETHDestChainToBaseChain}{Sequence Diagram: Bridge Native Crypto from \textit{Dest Chain} to \textit{Base Chain}}{1.}

% bridgeETHRollback
\customfigure{problem_approach/flows/bridgeETHRollback}{Activity Diagram: Rollback on bridge Native Crypto failure}{1.}

\subsubsection{NFTs Transfer}

% bridgeNFTBaseChainToDestChain
\customfigure{problem_approach/flows/bridgeNFTBaseChainToDestChain}{Sequence Diagram: Bridge NFT from \textit{Base Chain} to \textit{Dest Chain}}{1.}

% bridgeNFTDestChainToBaseChain
\customfigure{problem_approach/flows/bridgeNFTDestChainToBaseChain}{Sequence Diagram: Bridge NFT from \textit{Dest Chain} to \textit{Base Chain}}{1.}

% bridgeNFTRollback
\customfigure{problem_approach/flows/bridgeNFTRollback}{Activity Diagram: Rollback on bridge NFTs failure}{1.}


\subsubsection{Generic transactions}

% bridgeGenericDestChainToDestChain
\customfigure{problem_approach/flows/bridgeGenericDestChainToDestChain}{Sequence Diagram: Bridge Generic TX}{1.}

% bridgeGenericRollback
\customfigure{problem_approach/flows/bridgeGenericTXRollback}{Activity Diagram: Rollback on bridge TX failure}{1.}

