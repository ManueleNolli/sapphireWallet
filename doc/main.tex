%% Esempio per lo stile supsi
\documentclass[twoside]{packages/supsistudent}

% per settare noindent
\setlength{\parindent}{0pt}


% Crea un capitolo senza numerazione che pero` appare nell'indice %
\newcommand{\problemchapter}[1]{%
  \chapter*{#1}%
  \addcontentsline{toc}{chapter}{#1}%
\markboth{#1}{#1}
}

% Numerazione delle appendici secondo norma
\addto\appendix{
\renewcommand{\thesection}{\Alph{chapter}.\arabic{section}}
\renewcommand{\thesubsection}{\thesection.\arabic{subsection}}}

\setcounter{secnumdepth}{5} 	%per avere più livelli nei titoli
\setcounter{tocdepth}{5}		%per avere più livelli nell'indice


\doctitle{Blockchain Interoperability}
\student{Nolli Manuele}
\supervisor{Gremlich Giuliano}
\cosupervisor{Guidi Roberto}
\course{Master of Science in Engineering: Computer Science}
\anno{2023 - 2024}

% Image
\graphicspath{ {./images/} }


\begin{document}

\pagenumbering{alph}
\maketitle
\onehalfspacing
\frontmatter


\pagenumbering{roman}
\tableofcontents
\listoffigures					% Opzionale
\listoftables					% Opzionale

\newpage
\mainmatter
\pagenumbering{arabic}
\setcounter{page}{1}

\problemchapter{Abstract}
\label{chap:abstract}

This paper presents the specialization project by Manuele Nolli, a student of the Master of Science in Engineering (MSE) in Switzerland with a focus on Computer Science. The project, in collaboration with the Institute of Information Systems and Networking (ISIN) at the University of Applied Sciences and Arts of Southern Switzerland (SUPSI), aims to analyse the current state of Blockchain technology, specifically in the areas of Account Abstraction and Cross-Chain Communication. The core of the project is the development of a decentralized wallet, named Sapphire Wallet, which allows users to interact seamlessly with multiple blockchains without requiring technical knowledge.

The results demonstrate that Sapphire Wallet significantly improves user interface and experience compared to traditional Ethereum Virtual Machine (EVM) networks, offering intuitive transaction processes and enhanced account recoverability through trusted third parties (Guardians). However, the implementation faces challenges such as slow transaction speeds and high deployment costs, rendering it impractical for global adoption in its current form.

Despite these limitations, the Sapphire Wallet showcases the potential of the Account Abstraction paradigm and highlights the necessity for protocol updates to fully implement its capabilities. The paper also suggests that new blockchain systems, like ICP, which incorporate Account Abstraction and advanced Cross-Chain Communication paradigms, should be closely observed as they might represent the future of blockchain technology.


\problemchapter{Introduction}
\label{chap:introduction}

Introduction
\chapter{Motivation and Context}
\label{chap:motivation_and_context}
\chapter{Problem}
\label{chap:problem}

Blockchain technology lacks of \textit{user experience}, every time a user wants to interact with a Blockchain, he needs to sign a transaction, pay a fee, and wait for the transaction to be confirmed. This is a big barrier for the adoption of Blockchain technology, compared to the traditional web 2.0 applications where the interaction is seamless, and the user does not need to know anything about the underlying technology. 

The field is a very active research area, many solutions and chains have been proposed. In fact, millions of users and developers started to use them. \cite{blockchain-statistics} However, an important question arises when there are multiple chains: \textit{which one should I use?}

Interoperability is the answer to this question, as it allows different chains to communicate with each other. In this way, the decision of which chain to use is not relevant and a business can benefit from the best features of each chain without losing market share. But how can this be achieved \textit{without experienced users?}

The goal of this Master project is to analyse the current state of the art of Blockchain \textit{Interoperability} and \textit{Account Abstraction}, and to propose a solution that allows users to interact with multiple chains without the need to have any kind of knowledge about the used technology.
\chapter{State of the Art}
\label{chap:state_of_the_art}

This chapter presents the state of the art of blockchain interoperability and account abstraction, with an emphasis on EVM-compatible blockchains.\cite{evm} In the first section, the concept of account abstraction is introduced, with a focus on the evolution of the ERC standards and the Argent wallet. The second section presents the state of the art of cross-chain communication, categorising the bridges according to the trust model, communication model, and asset transfer model. Finally, the Internet Computer Protocol (ICP) is presented, as it is a promising solution for both account abstraction and cross-chain communication. 


\section{Account Abstraction}
\label{sec:account_abstraction}

% Different solutions

Different approaches have been proposed to improve the user experience (UX) of web3.0 applications, including \textit{Meta Transactions}, \textit{Smart Contract Wallets}, and \textit{Account Abstraction}. The results of the analysis done by the Institute of Information Systems and Networking at the University of Applied Sciences and Arts of Southern Switzerland (SUPSI) indicate that \textit{Account Abstraction} presents the most promising approach for improving UX in decentralised systems, offering a comprehensive solution that enables smart contract logic to determine the fee payment and validation logic of transactions. \cite{isin-aa-user-experience}


%What is an account?
To analyse the concept of Account Abstraction, it is necessary to understand what an account is in the context of Ethereum, there are two types of accounts: \cite{ethereum-accounts}
\begin{itemize}
    \item \textit{Externally Owned Accounts (EOA)}: Controlled by anyone with the private key.
    \item \textit{Contract Accounts}: A deployed smart contract. Controlled by the code.
\end{itemize}

% Problem with EOA: fee, lost private key, etc.

EOAs are the only account type that can initiate transactions, while contract accounts can only \textit{react}\footnote{A smart contract can initiate a transaction by calling another smart contract, but the transaction is still initiated by an EOA.} to transactions. The main problem with EOAs is that they are not flexible enough to support the needs of modern applications. For instance, EOAs can not be used to initiate batches of transactions and requires to always keep an ETH balance to cover gas. Other limitations include the lack of recovery options, the need to pay a fee for each transaction, and the risk of losing the private key. With these weaknesses, the user experience is not as seamless as in traditional web2.0 applications. \cite{ethereum-account-abstraction}

% What is account abstraction? Use a contract accounts instead of EOAs.
% Why use account abstraction? different use cases: recovery, relaying transactions, etc.

\textit{Account abstraction} is a way to solve these problems by allowing users to flexibly include more security and better user experiences into their accounts. This can happen in three ways: \cite{ethereum-account-abstraction}
\begin{itemize}
    \item \textit{Upgrading EOAs}: So that they can be controlled by Smart Contracts.
    \item \textit{Upgrading Smart Contracts}: So that they can initiate transactions. 
    \item \textit{Separate the transaction system}: Adding a second and separate transaction system.
\end{itemize}

The first two solutions require an upgrade of the Ethereum protocol, while the third solution can be implemented without changing it.\cite{ethereum-account-abstraction} All possible paths will be discussed in the following sections.

Regardless of the route, the outcome is access to Ethereum via \textit{Smart Contract Wallets}, either natively supported as part of the existing protocol or via an add-on transaction network.  \cite{ethereum-account-abstraction}

The use of \textit{Smart Contract Wallets}  unlocks new possibilities, including: \cite{ethereum-account-abstraction}
\begin{itemize}
    \item \textit{Flexible security models}: Multi-signature, account freezing, transaction limits, allowlists, etc.
    \item \textit{Recovery options}: Social recovery, guardians, etc.
    \item \textit{Multi owner accounts}: Shared accounts, business accounts, etc.
    \item \textit{Relayed transactions}: Sign a transaction and let someone else pay for it.
    \item \textit{Batch/Multi transactions}: Execute multiple transactions in a single call.
    \item \textit{Innovative User Experience}: Seamless interaction with the Blockchain.
\end{itemize}

% Write something to introduce EIP and Argent

In the following sections, the most relevant Ethereum Improvement Proposals (EIPs) related to Account Abstraction are presented. In particular, it is possible to notice how the community is advancing different approaches to achieve the same goal. Moreover, the Argent project is presented as an example of a real-world implementation of Account Abstraction.

\subsection{EIPs}
\label{subsec:eips}

The community of Ethereum is actively working to improve the Blockchain. Everyone can propose an improvement through an EIP (Ethereum Improvement Proposal). An EIP is a design document providing information to the Ethereum community, or describing a new feature for Ethereum or its processes or environment. The EIP should provide a concise technical specification of the feature and a rationale for the feature. The EIP author is responsible for building consensus within the community and documenting dissenting opinions. \cite{eip-1}

In the context of Account Abstraction, the most relevant EIPs are in the category of \textit{Standards Track EIPs}, which are changes that affect most or all Ethereum implementations. Standards Track EIPs can also be classified into four categories: \cite{eip-1}
\begin{itemize}
    \item \textit{Core}: Changes that require a consensus (hard) fork.
    \item \textit{Networking}: Changes that affect network protocol, including changes that are specific to the network layer.
    \item \textit{Interface}: Includes improvements around language-level standards, like method names and contract ABIs.
    \item \textit{ERC}: Application-level standards and conventions, including contract standards such as token standards (ERC-20), name registries (ERC-137), URI schemes, library/package formats, and wallet formats.
\end{itemize}

In the following sections, the most relevant EIPs related to Account Abstraction are presented. The EIPs are ordered by the date of the last update, from the oldest to the newest.

\subsubsection{EIP-86}
\label{subsubsec:eip-86}

Proposed by Vitalik Buterin\footnote{co-founder of Ethereum} in 2017, EIP-86 is a \textit{Core} proposal and can be considered the first step towards Account Abstraction. The goal of this proposal is to abstract the signature verification and the nonce scheme. This allows to develop smart contracts able to use any signature scheme in transaction verification. \cite{eip-86}

The default way to secure an Ethereum account is the ECDSA\footnote{Elliptic Curve Digital Signature Algorithm \cite{ECDSA}} signature scheme. However, EIP-86 allows to use any signature scheme, including multi-signature schemes and \textbf{custom cryptography}. \cite{eip-86}

It is important to notice that ECDSA cryptography is \textbf{not} quantum-resistant, the Peter Shor's algorithm can be used to break the ECDSA in polynomial time. \cite{stewart2018committing}

The current status of EIP-86 is \textit{Stagnant}, as it has been inactive for more than 6 months. Nevertheless, this proposal was an inspiration for the following EIPs.

\subsubsection{EIP-2711} %Remove ???
\label{subsubsec:eip-2711}

Proposed in 2020, The EIP-2771 is based on the EIP-2718 proposal, which introduces a new transaction type. The main features that EIP-2771 may introduce are: \cite{eip-2711} 

\begin{itemize}
    \item \textit{Sponsored Transactions}: Allow third parties to pay for a user's gas costs.
    \item \textit{Batch Transactions}: Execute multiple transactions in a single call.
    \item \textit{Expiring Transactions}: Set an expiration time that invalidates the transaction.
\end{itemize}

To achieve these features, the concept of \textbf{meta-transaction} has been introduced. The idea is that transactions signed by a user get sent to a \textit{Forwarder} contract. The forwarder is a trusted \textbf{off-chain} entity that verifies that transactions are valid before sending them on to a gas relay. The forwarder passes the transaction on to a Recipient contract, paying the necessary gas to make the transaction executable on Ethereum. The transaction is executed if the Forwarder is known and trusted by the Recipient. This model makes it easy for developers to implement gasless transactions for users. \cite{ethereum-account-abstraction}

The proposal is a \textit{Core} EIP and it has been \textit{Withdrawn} by the author. The reason for the withdrawal is that the EIP-3074, proposed with the help of the EIP-2711 author, is a more complete solution. \cite{eip-2711}

\subsubsection{EIP-2938}
\label{subsubsec:eip-2938}

Proposed in 2020, The EIP-2938 is a \textit{Core} proposal that introduces a new EVM opcode\footnote{An Opcode is a machine-level instruction} called \texttt{PAYGAS}. As \hyperref[subsubsec:eip-2711]{EIP-2711}, also EIP-2938 is based on EIP-2718.

The main concept of EIP-2938 is to remove intermediaries during a relayed transaction. The implementations that need a Relayer are technically inefficient, due to the extra 21000 gas to pay for the relayer, economically inefficient, as relayers need to make a profit on top of the gas fees that they pay. Additionally, use of intermediary protocols means that these applications cannot simply rely on base Ethereum infrastructure and need to rely on extra protocols that have smaller userbases and higher risk of no longer being available at some future date. \cite{eip-2938}

Also the EIP-2938 is in a \textit{Stagnant} status, as it has been inactive for more than 6 months. The community is currently favouring EIP-4337. \cite{ethereum-account-abstraction}

\subsubsection{EIP-3074}
\label{subsubsec:eip-3074}

Proposed in 2020, EIP-3074 introduces two new opcodes to the Ethereum Virtual Machine (EVM): \texttt{AUTH} and \texttt{AUTHCALL}. The first sets a context variable authorized based on an ECDSA signature. The second sends a call as the authorized account. This essentially delegates control of the externally owned account (EOA) to a smart contract called \textit{Invoker}. The main purpose of EIP-3074 is to \textbf{upgrade EOAs}. \cite{eip-3074}

More specifically, \texttt{AUTH} enables the retrieval of a user's address from a signed message, effectively authenticating the user within EVM. \texttt{AUTHCALL}, on the other hand, simplifies the execution of authenticated calls by altering the sender of the transaction to the authorized address, thus simplifying the process of executing transactions that mimic smart contract logic directly from EOAs. \cite{ethereum-account-abstraction}

This gives developers a flexible framework for developing novel transaction schemes for EOAs. A motivating use case of this EIP is that it allows any EOA to act like a smart contract wallet without deploying a contract (many EOAs may use the same \textit{Invoker} contract). \cite{eip-3074}

The \textit{Invoker} contract is a trustless intermediary between the EOA and the rest of the Ethereum network. The contract can be used to implement a variety of features, such as \textit{sponsored transactions}, \textit{expiration}, \textit{batch transactions}, etc. \cite{eip-3074}

As stated in the EIP: \cite{eip-3074}
\begin{displayquote}
    Choosing an invoker is similar to choosing a smart contract wallet implementation. It's important to choose one that has been thoroughly reviewed, tested, and accepted by the community as secure. We expect a few invoker designs to be utilized by most major transaction relay providers, with a few outliers that offer more novel mechanisms.
\end{displayquote}

EIP-3074 is currently in \textit{Review} status, and it is expected to be included in the next hard fork of Ethereum, named \textit{Pectra}, expected in Q4 2024. \cite{pectra-hardfork}

\subsubsection{ERC-4337}
\label{subsubsec:erc-4337}

Proposed in 2021, ERC-4337 is an Account Abstraction proposal which completely avoids the need for consensus-layer protocol changes by creating a \textbf{separate transaction system}. \cite{eip-4337}

Instead of adding new protocol features and changing the bottom-layer transaction type, this proposal introduces a higher-layer pseudo-transaction object called a \textit{UserOperation}. 
As illustrated by figure \ref{fig:state_of_the_art/4337-diagram}, users send \textit{UserOperation} objects into a separate \textit{mempool}. A special class of actor called bundlers package up a set of these objects into a transaction making a \textit{handleOps} call to a special contract, and that transaction then gets included in a block. 

\customfigure{state_of_the_art/4337-diagram}{ERC-4337 Diagram}{0.8}

ERC-4337 also introduces a paymaster mechanism that can enable users to pay gas fees using ERC-20 tokens (e.g. USDC) instead of ETH or to allow a third party to sponsor their gas fees altogether, all in a decentralised fashion. \cite{isin-aa-user-experience}

The ERC-4337 is in \textit{Draft} status and the main start contract required was deployed in March 2023. \cite{ethereum-roadmap-ux}

\subsubsection{EIP-7702}
\label{subsubsec:erc-7702}

Proposed in 2024 co-authored by Vitalik Buterin, EIP-7702 is the latest proposal for Account Abstraction. The main goal of this proposal is to offer the same features as \hyperref[subsubsec:eip-3074]{EIP-3074}, but without the need of the two opcodes \texttt{AUTH} and \texttt{AUTHCALL}. The reason is that, as stated in the EIP, the two opcodes will not be used in the final solution for Account Abstraction, where Smart Contract Wallets will eventually be implemented. \cite{eip-7702}

EIP-7702 propose to a new transaction type that adds a \texttt{contract\_code} field and signature, the idea is to convert the signing account (not necessarily the same that started the transaction) into a smart contract wallet for the duration of that transaction. \cite{eip-7702}

This \textit{Core} proposal is in an early stage, as it is in \textit{Draft} status. But it is promising, as it is co-authored by Vitalik Buterin, one of the co-founders of Ethereum.

\subsection{Argent}
\label{subsec:argent}

Argent\footnote{https://www.argent.xyz/} is a well known company whose mission is to make Blockchain widely used. To achieve this goal, Argent, as a pioneer in the field of Account Abstraction, has developed a wide range solutions to improve the user experience of Blockchain applications. \cite{argent-aa}

One of the first solution developed by Argent is the open source project \textit{Argent Wallet Smart Contract}. The Argent wallet is an Ethereum Smart Contract based mobile wallet. The wallet's user keeps an EOA secretly on his mobile device. This account is set as the owner of the Smart Contract. User's funds (ETH and ERC20 tokens) are stored on the Smart Contract. With that model, any kind of logic can be added to the wallet to improve both the user experience and the wallet security. \cite{argent-github}

% Argent does not use the EIPs, because they are not yet implemented in the Ethereum mainnet. but with a more complex solution, they have achieved the same goal.

Argent smart contracts do not use the \hyperref[subsec:eips]{EIPs} presented in the previous section, as the major of them need a hard fork to be implemented in the Ethereum Mainnet. Instead, Argent has developed a more complex solution to achieve the same goal. 

% Argent functionalities

The functionalities of the Argent wallet are: \cite{argent-smart-wallet-specifications}
\begin{itemize}
    \item \textit{Guardians}: Authorized accounts (EOA or smart contracts) permitted by the wallet owner to execute specific operations such as locking, unlocking, initiating recovery procedures, approving transactions to unknown accounts, or creating session keys. Guardians can be various entities including friend's wallets, EOAs, hardware wallets, or paid third-party services. Adding or removing guardians requires confirmation within a specified time frame to prevent unauthorized access.
    \item \textit{Recovery}: Mechanism allowing wallet owners to regain access to their funds in case of loss or compromise, typically involving a predefined recovery process facilitated by designated guardians. Recovery processes require validation by the wallet's guardians and can be cancelled or finalized within a certain time frame.
    \item \textit{Ownership Transfer}: Allows users to transfer ownership of their wallet to a new device while still in possession of the original device. The transfer is immediate to avoid service interruption but requires approval from guardians.
    \item \textit{Multi-Call}: Enables the wallet to interact with the Ethereum ecosystem by executing a sequence of transactions one after the other. The multi-call fails if any of the transactions fail and requires specific conditions to be met for execution.
    \item \textit{Trusted Contacts}: Allows users to maintain a list of trusted addresses managed by the wallet owner. Interacting with trusted contacts, such as sending assets, does not require additional authorization.
    \item \textit{Dapp Registry}: Supports two registries of authorized addresses, including native integration dapps and curated dapps accessible through WalletConnect. Entries in the registry are time-locked and require guardian approvals for enabling or disabling.
    \item \textit{Session}: Users can execute multiple transactions within a given time window with guardian approval, required only once by creating a session with a temporary session key.
    \item \textit{Upgradability}: Allows for the wallet to be upgraded to add new features or fix potential bugs, with the choice left to the wallet owner without the ability for a centralized party to force an upgrade.
    \item \textit{ETH-Less Account}: Enables owner and guardians to execute wallet operations without owning ETH or paying transaction fees by utilizing \textit{Meta Transactions}, where a third-party relayer executes transactions on their behalf.
    \end{itemize}
    
\subsubsection{Argent Wallet Smart Contract Architecture}

The simplified architecture of the Argent Wallet Smart Contract is shown in attachment \addattachment{ArgentWalletsBlockchainSmartContracts.png}{ArgentWalletsBlockchainSmartContracts}, where the main components are: \cite{argent-smart-wallet-specifications}

\begin{itemize}
    \item \textit{Wallet Factory}: Wallet factory contract used to create proxy wallets using \texttt{CREATE2}\footnote{Gives the ability predict the address where a contract will be deployed, without ever having to do so. \cite{opcode-create2}} opcode and assign them to users.
    \item \textit{Base Wallet and Proxy}: The Base Wallet is a simple library-like contract implementing basic wallet functionalities that are not expected to ever change. These functionalities include changing the owner of the wallet, (de)authorizating modules and performing (value-carrying) internal transactions to third-party contracts. The Proxy Wallet is a contract that delegates all calls to the Base Wallet, the goal is to reduce the deployment cost of each new wallet.
    \item \textit{Module Registry}: The Module Registry maintains a list of the registered Module contracts that can be used with user wallets.
    \item \textit{Argent Module}: All the functionalities of the Argent wallet are currently implemented in a single module. To structure the code, the \textit{ArgentModule} inherits from several abstract managers that each implements a coherent subset of the logic:
        \begin{itemize}
            \item \textit{Transaction Manager}: Manages the execution of multi-calls, adding or removing trusted contacts and session executions.
            \item \textit{Security Manager}: Manages the wallet's security features, including guardians, recovery, locking and ownership transfer.
            \item \textit{Relayer Manager}: Entry point for Meta Transactions, allowing users to execute transactions without owning ETH.
        \end{itemize}
    \item \textit{Dapp Registry}: The Dapp Registry is a contract that maintains a list of dapps that are allowed to interact with the wallet.
    \item \textit{Storages}: The wallet's state is stored in two separate storage contracts:
    \begin{itemize}
        \item \textit{Transfer Storage}: Used to store the trusted contacts of a wallet.
        \item \textit{Guardian Storage}: Used to store the list of guardians of a wallet.
    \end{itemize}
\end{itemize}
\section{Cross-Chain Communication}
\label{sec:cross_chain_communication}

This section presents the state of the art of \textit{Cross-Chain communication}, categorising the bridges according to the trust model, communication model, and asset transfer model. The need for Cross-Chain communication arises from the fragmentation of the blockchain ecosystem, where each blockchain is a separate network with its own consensus mechanism and state. The goal of Cross-Chain communication is to enable the transfer of assets and information between different blockchains, allowing users to interact with multiple network seamlessly. 

\subsection{Bridge classification}
\label{subsec:bridge_classification}

Cross-chain bridges can be classified based on various criteria, each highlighting different aspects of their functionality and security. Understanding these classifications helps in assessing the suitability of a bridge for specific applications and in recognizing the trade-offs involved. This section categorises bridges according to their trust model, communication model, and asset transfer model, providing a comprehensive overview of the current landscape in cross-chain technology. \cite{lifi-bridge}

\subsubsection{Trust model}
\label{subsubsec:trust_model}

Blockchain bridges can be classified into two main trust models: 
\begin{itemize}
    \item \textit{Trusted Bridges} rely on a central authority to facilitate Cross-Chain transactions. \textit{Users must trust this authority} to manage their funds securely. These bridges are generally easier to implement but come with the risk of centralization, which can lead to single points of failure and security vulnerabilities. \cite{immunebytes-trust-bridge} An example is \textit{Multichain}\footnote{https://multichain.xyz/}.
    \item \textit{Trustless Bridges} eliminate the need for a central authority by using smart contracts and \textit{cryptographic algorithms}. These bridges ensure users retain control over their assets throughout the transaction process. However, they can be complex to design and are susceptible to smart contract bugs. \cite{immunebytes-trust-bridge} Trustless bridges are in a early stage of development, with projects like \textit{EOS Network}\footnote{https://eosnetwork.com/}.
\end{itemize}

\subsubsection{Communication model}
\label{subsubsec:communication_model}

The communication model of blockchain bridges can be categorised based on what type of blockchain interaction they enable:  \cite{lifi-bridge}
\begin{itemize}
    \item \textit{L1 $\leftrightarrow$ L1 Bridges}: Connect different Layer 1 blockchains with each other. For example, the Avalanche Bridge\footnote{https://core.app/bridge} connects Ethereum and Avalanche.
    \item \textit{L1/L2 $\leftrightarrow$ L2 Bridges}: Connect the Mainnet with different Layer 2 solutions and the L2s with each other. Arbitrum\footnote{https://arbitrum.io/} is an example of a Layer 2 solution that can be connected to the Ethereum Mainnet.
\end{itemize}

\subsubsection{Asset transfer model}
\label{subsubsec:asset_transfer_model}

The asset transfer mechanisms of blockchain bridges include:  \cite{lifi-bridge}  \cite{chainlink-transfer-assets}
\begin{itemize}
    \item \textit{Lock \& Mint}: Assets are locked on the source blockchain and equivalent tokens are minted on the destination blockchain. 
    \item \textit{Burn \& Mint}: Assets on the source blockchain are burned, and equivalent tokens are minted on the destination blockchain.  
    \item \textit{Lock \& Unlock}: Tokens are locked on the source chain and unlocked from a liquidity pool on the destination chain. 
\end{itemize}




\section{IC: Internet Computer}
\label{sec:icp}

The Internet Computer (IC)\footnote{https://internetcomputer.org/} is a blockchain protocol developed by the DFINITY Foundation\footnote{https://dfinity.org/}, a Switzerland non-profit organization.  

IC's vision is that most of the world's software will be replaced by smart contracts. To realize that vision, IC is designed to make smart contracts as powerful as traditional software. \cite{icp-vision}

\subsection{Architecture}

The Internet Computer (IC) realizes the vision of a \textit{World Computer}, an open and secure blockchain-based network that can host programs and data in the form of smart contracts, perform computations on smart contracts in a secure and trustworthy way, and scale infinitely.\cite{icp-how-it-works}

Smart contracts on the Internet Computer are called \textbf{Canister} smart contracts, each consisting of a bundle of WebAssembly (Wasm) bytecode and storage. Each canister has its own, isolated, data storage that is only changed when the canister executes code. \cite{icp-how-it-works}

The IC is designed to be highly scalable and efficient in terms of hosting and executing canister smart contracts. The top-level building blocks of the IC are subnetworks, or \textbf{subnets}, the IC as a whole consists of many subnets, where each subnet is its own blockchain that operates concurrently with and independently of the other subnets (but can communicate asynchronously with other subnets). Each subnet hosts canister smart contracts, up to a total of hundreds of gigabytes of replicated storage. A subnet consists of node machines, or nodes. Each node replicates all the canisters, state, and computation of the subnet using blockchain technology. This makes a subnet a blockchain-based replicated state machine, that is, a virtual machine that holds state in a secure, fault-tolerant, and non-tamperable manner. The computations of the canisters hosted on a subnet will proceed correctly and without stopping, even if some nodes on the subnet are faulty (either because they crash, or even worse, are hacked by a malicious party). New subnets can be created from nodes added to the IC to scale out the protocol, analogous to how traditional infrastructures such as public clouds scale out by adding machines. \cite{icp-architecture}

\textcolor{red}{Add figure}

\subsubsection{IC Protocol (ICP)}

\textcolor{red}{Go deeper or remove?}

The Internet Computer is created by the Internet Computer Protocol (ICP). The ICP is a 4-layer protocol that is running on the nodes of each subnet. The 4 layers are: \cite{icp-how-it-works}

\begin{itemize}
    \item \textit{Peer-to-peer}: Responsible for the secure and reliable communication between the nodes of a subnet. Using P2P, a node can broadcast a network message to all the nodes in the subnet.
    \item \textit{Consensus}: Allows the nodes to agree on the messages to be processed, as well as their ordering. Consensus is the component of the core IC protocol that drives the subnets of the IC.
    \item \textit{Message routing}: Receives a block of messages to be processed from consensus and places the messages into the input queues of the target canisters.
    \item \textit{Execution}: execute canister smart contract code.
\end{itemize}


\subsection{Internet Identity: Account Abstraction}

Internet Identity is the ICP's native form of identity. IC uses \textit{passkeys}, a unique public/private key pair stored in the secure hardware chip, to authenticate users. They offer a convenient and more secure alternative to passwords, enabling a password-free login experience for websites and applications. \cite{icp-identity}

Passkeys are based on the WebAuthentication (WebAuthn) cryptographic standard. When creating an account, the operating system generates a unique and app- or website-specific cryptographic key pair \textit{on the device}. \cite{icp-passkey}

To interact with a canister, users need to authenticate their actions using a cryptographic key pair (passkey). The network verifies the digital signature using the associated public key, ensuring that the action was indeed signed by the authorized user. If the verification is successful, the action is executed by the canister on the Internet Computer. \cite{icp-passkey}

Internet Identity offers several benefits over other blockchains account systems, some \hyperref[sec:account_abstraction]{\textit{Account Abstraction}} features are natively supported: \cite{icp-identity-technical}
\begin{itemize}
    \item \textit{Improved User Experience}: Authentication via cryptographic key on the device without the need of passwords.
    \item \textit{Security}: Users can add many devices to the same identity \textit{Anchor}\footnote{An anchor is a number attached to a user's identity. This number is auto-generated by the system and there is only one per identity.}. A common example is that users may add their smartphone and desktop computer to an identity anchor. 
    \item \textit{Recovery Mechanism}: There are two recovery mechanisms for recovery the \textit{Anchor}:
        \begin{itemize}
            \item \textit{Seed Phrase}: A user can select this option to generate a cryptographically-secure seed phrase that they can use to recover an Identity Anchor. 
            \item \textit{Security Key}: A dedicated security key can be used to recover an Identity Anchor in the event that a user loses access to their authorized devices. 
        \end{itemize}
\end{itemize}

\subsection{Chain-key Cryptography: Interoperability}

\textit{Chain-key cryptography} enables subnets of the Internet Computer to jointly hold cryptographic keys, in a way that no small subset of potentially misbehaving nodes on the subnet can perform useful operations with the key, but the majority of honest nodes together can. \cite{icp-chain-key}

One of the major benefits of chain-key cryptography is that it enables canister to \textit{natively create signed transactions on other blockchains} like Ethereum or Bitcoin \textit{without a bridge}. \cite{icp-cross-chain-interoperability} With the combination of HTTPS outcalls feature and chain-key cryptography, the Internet Computer canisters can interact with external services and blockchains. \cite{icp-https-outcalls}

For instance, ICP is integrated with the Bitcoin network using chain-key ECDSA signatures and a protocol-level integration, allowing for a canister to create a Bitcoin address, then send or receive bitcoin directly as if they were a regular Bitcoin user. ICP will also use chain-key ECDSA to facilitate an upcoming integration with Ethereum that will allow Ethereum smart contracts and digital assets like ERC-20 tokens to be used in ICP canisters. \cite{icp-cross-chain-interoperability}


\chapter{Problem Approach: Sapphire Wallet}
\label{chap:problem_approach}

In this chapter, The Sapphire Wallet ecosystem is presented. The Sapphire Wallet is a Multi-Chain Account Abstraction, based on an improved version of the \hyperref[subsec:argent]{Argent Wallet}.

Sapphire Wallet is a Proof of Concept of the future of blockchain technology. The main goal of Sapphire Wallet is to give an idea of the potential of the blockchain technology in combination with Account Abstraction Layer and Multi-Chain Bridge approach. To achieve this goal, the Sapphire Wallet includes a set of features already present in the Argent Wallet, such as guardian, wallet creation and wallet recovery. Moreover, the Sapphire Wallet ecosystem includes a Multi-Chain Bridge, which allows the user to interact with different Blockchains. 

It is important to notice that the Sapphire Wallet is developed on the current version of EVM\footnote{Dencun version}. There are several promising \hyperref[subsec:eips]{EIPs} that could be implemented in the future, such as the \hyperref[subsubsec:eip-3074]{EIP-3074} and the \hyperref[subsubsec:erc-7702]{EIP-7702}, but they need an hard fork of the Ethereum network. Whatever the final Account abstraction implementation is, the Sapphire Wallet provides a realistic Proof of Concept of how the users will interact with the Blockchain in the future.

In the following sections, the architecture and the components of the Sapphire Wallet ecosystem are presented. Each section is described in isolation, but at the end of the chapter, the operational flow and the use cases are presented to give a complete overview of the Sapphire Wallet functionalities.

\section{Architecture}
\label{sec:architecture}

The Sapphire Wallet ecosystem is composed of four components:

\begin{itemize}
    \item \textit{Blockchain}: Two or more Blockchains networks of which one is called \textit{Base Chain} and the others are called \textit{Dest Chain(s)}. The Acccount Abstraction Layer is implemented on the \textit{Base Chain}.
    \item \textit{Bridge}: \textit{Lock \& Mint} bridge that allows the user to transfer assets and \textit{transactions} between the \textit{Base Chain} and the \textit{Dest Chain(s)}.
    \item \textit{Backend}: Infrastructure that simplifies the interaction between the Blockchains and the Mobile Application. Moreover, the Backend is the \textit{Relayer} that sends the transactions to the Blockchain networks.
    \item \textit{Mobile Application}: User interface that allows the user to interact with the Sapphire Wallet ecosystem.
\end{itemize}

A simplified architecture of the Sapphire Wallet Infrastructure is shown in Figure \ref{fig:problem_approach/infrastructure}. The specific components will be described in the following sections.

\borderfigure{problem_approach/infrastructure}{Simplified Sapphire Infrastructure}{1}
\section{Blockchain}
\label{sec:blockchain}

Various smart contracts are implemented  and deployed on the blockchain to manage the Sapphire Wallet. There is a distinction between the Base Chain and the Dest Chain(s):
\begin{itemize}
    \item \textit{Base Chain}: smart contracts to manage Account Creation, Wallet Recovery, Transaction execution, etc.
    \item \textit{Dest Chain(s)}: smart contracts to abstract the \textit{Base Chain} accounts and securely implement the bridge between the \textit{Base Chain} and the \textit{Dest Chain(s)}. 
\end{itemize}

\subsection{Base Chain}
\label{subsec:base_chain}

The Base Chain is the main blockchain where the Sapphire Wallet is deployed. The contracts are an extension of the \hyperref[subsec:argent]{Argent Wallet} contracts. 

\subsection{Dest Chain(s)}
\label{subsec:dest_chain(s)} 

\section{Bridge}
\label{sec:bridge}

\section{Backend}
\label{sec:backend}

The backend of the Sapphire Wallet is a set of microservices that delegate and simplify the interaction between the user and the blockchain. Moreover, the backend is the \textbf{Relayer} of the Sapphire Wallet. Indeed, the \hyperref[sec:mobile_application]{Mobile Application} will send the signed transactions to the backend, which will execute (and pay) them on behalf of the user.

As shown in the figure \ref{fig:problem_approach/backend_infrastructure}, the backend is composed of the following components:
\begin{itemize}
    \item \textit{API Gateway}: the entry point of the backend. It is responsible for the routing of the requests to the correct microservice.
    \item \textit{Wallet Factory}: the microservice that creates the user wallet on the \textit{Base Chain}.
    \item \textit{Sapphire Relayer}: the microservice that executes the transactions on behalf of the user.
    \item \textit{Sapphire Portfolio}: the microservice that retrieves the user wallet information on all the chains.
\end{itemize}

\customfigure{problem_approach/backend_infrastructure}{Sapphire Wallet Backend Infrastructure}{1}

The backend is developed in Typescript and uses the \textit{NestJS}\footnote{https://nestjs.com/} framework. The communication between the microservices is done via \textit{TCP} protocol to ensure a better performance. 

\subsection{API Gateway}
\label{subsec:api_gateway}

The API Gateway is the entry point of the backend. It is responsible for the routing of the requests to the correct microservice. The communication is done using NestJs Event-Based Pattern. 

\subsection{Wallet Factory}
\label{subsec:wallet_factory}

The Wallet Factory is the microservice that creates the user wallet on the \textit{Base Chain}. The user wallet is created using the \textit{WalletFactory} smart contract.  

The Microservice will wait until the transaction is fully confirmed on the blockchain before returning the wallet address to the user.

\subsection{Sapphire Relayer}
\label{subsec:sapphire_relayer}

The Sapphire Relayer has two main tasks:

\begin{itemize}
    \item \textit{Execute} the transactions on behalf of the user. The transactions are sent by the \hyperref[sec:mobile_application]{Mobile Application} and are signed by the user. The Relayer will execute the transaction and pay the gas fee.
    \item \textit{Authorise} new wallets address. Each Sapphire Wallet has a list of authorised wallets with which it can interact.
\end{itemize}

The Microservice will wait until the transaction is fully confirmed on the blockchain before returning a confirmation or an error to the user.

\subsection{Sapphire Portfolio}
\label{subsec:sapphire_portfolio}

In a Multi-Chains context, the retrieval of the wallet information is a complex task. Especially because the ownership of NFT and tokens are stored in different smart contracts.

The Sapphire Portfolio is the microservice that retrieves the wallet info on all the chains. The microservice will firstly retrieve the \textit{AccountContract} address of each \textit{Dest Chain} used by the user. Then, it will retrieve the NFTs and the tokens owned by the user on each chain. Finally, it will return the wallet info to the user.
\section{Mobile Application: Sapphire Wallet}
\label{sec:mobile_application}

\subsection{Wallet Creation}
\label{subsec:wallet_creation}

\subsection{Wallet Recovery}
\label{subsec:wallet_recovery}

\subsection{Home Page}
\label{subsec:wallet_home_page}

Write all the operations, balance of chains, button.

\subsection{NFTs Page}
\label{subsec:nfts_page}

\subsection{Settings Page}
\label{subsec:settings_page}

guardian


\section{Operational Flow and Use Cases}

\subsection{Base Chain}

\subsubsection{Wallet Creation}

% WalletCreation
\customfigure{problem_approach/flows/walletCreation}{Sequence Diagram: Wallet Creation}{0.8}

\subsubsection{Wallet Recovery}

% WalletRecovery
\customfigure{problem_approach/flows/walletRecovery}{Sequence Diagram: Wallet Recovery}{0.8}

% improvedWalletRecovery
\customfigure{problem_approach/flows/improvedWalletRecovery}{Sequence Diagram: Improved Wallet Recovery}{0.8}

\subsubsection{General Flow}

% TransactionsExecution
\customfigure{problem_approach/flows/transactionsExecution}{Sequence Diagram: General Transaction execution}{0.8}


% Explain that this works for crypto transfer and for general transactions (e.g. smart contracts calls).


\subsection{Portfolio}

\subsubsection{Balance}

% portfolioBalance
\customfigure{problem_approach/flows/portfolioBalance}{Sequence Diagram: Retrieve Native crypto balance in a Multi-Chain environment}{0.8}

\subsubsection{NFTs}

% portfolioNFTs
\customfigure{problem_approach/flows/portfolioNFTs}{Sequence Diagram: Retrieve owned NFTs in a Multi-Chain environment}{0.8}

\subsection{Multi-Chain}

\subsubsection{Native Crypto Transfer}

% bridgeETHBaseChainToDestChain
\customfigure{problem_approach/flows/bridgeETHBaseChainToDestChain}{Sequence Diagram: Bridge Native Crypto from \textit{Base Chain} to \textit{Dest Chain}}{1.}

% bridgeETHDestChainToBaseChain
\customfigure{problem_approach/flows/bridgeETHDestChainToBaseChain}{Sequence Diagram: Bridge Native Crypto from \textit{Dest Chain} to \textit{Base Chain}}{1.}

% bridgeETHRollback
\customfigure{problem_approach/flows/bridgeETHRollback}{Activity Diagram: Rollback on bridge Native Crypto failure}{1.}

\subsubsection{NFTs Transfer}

% bridgeNFTBaseChainToDestChain
\customfigure{problem_approach/flows/bridgeNFTBaseChainToDestChain}{Sequence Diagram: Bridge NFT from \textit{Base Chain} to \textit{Dest Chain}}{1.}

% bridgeNFTDestChainToBaseChain
\customfigure{problem_approach/flows/bridgeNFTDestChainToBaseChain}{Sequence Diagram: Bridge NFT from \textit{Dest Chain} to \textit{Base Chain}}{1.}

% bridgeNFTRollback
\customfigure{problem_approach/flows/bridgeNFTRollback}{Activity Diagram: Rollback on bridge NFTs failure}{1.}


\subsubsection{Generic transactions}

% bridgeGenericDestChainToDestChain
\customfigure{problem_approach/flows/bridgeGenericDestChainToDestChain}{Sequence Diagram: Bridge Generic TX}{1.}

% bridgeGenericRollback
\customfigure{problem_approach/flows/bridgeGenericTXRollback}{Activity Diagram: Rollback on bridge TX failure}{1.}


\chapter{Results}
\label{chap:results}

%%%%%%%%%
% WITH SAPPHIRE
%%%%%%%%%

% Deployment of contracts: 
% https://sepolia.etherscan.io/tx/0xeca9946ce552bcdaf05bebc7e0708e0a83fb4b35a6c40e3cfbcf1298f5836fa0
% https://sepolia.etherscan.io/tx/0xfc5e5ad4553769e5e95985d568ee6417a9ed77222afda2b9f4aef74a25b28c06
% https://sepolia.etherscan.io/tx/0xfead6d10fe5190a3e6a85fa3d3c384ed6ae196db0b154ae0fb04c627931c3bab
% https://sepolia.etherscan.io/tx/0xa1a74bbafacbf28ac3d1d41d65ba5b8cbb9c6f2dc8ef9873a776b62b19a550fc
% https://sepolia.etherscan.io/tx/0x8cb2d7c617f3bb2d1e0cfe0b8fc2c87051e05e7ec060f876441de83c493260d0
% https://sepolia.etherscan.io/tx/0x2164279fbd3bd5c7305f269af0c5823447f4aa09d978f9b8a0252438bc0733e5
% https://sepolia.etherscan.io/tx/0xdc1455a9f9dcbd99c3a97fe8b1b4f9f2e9c7090347f58f3ab0f94301f619cb34
% https://sepolia.etherscan.io/tx/0xc34b7037aa27c4c0b775d6e09f1f489ef9904158f747eab723f013acf771f04d
% https://sepolia.etherscan.io/tx/0xf7801500c8e2761e4ffad8d84a06ba8fd2f611c0b832af7512968d53b6443193
% total eth to deploy: 0.12872771
% https://www.oklink.com/amoy/tx/0x43ae451a87a7ebd40c70f55554accf690bd7da81ebc35fe7ce234de7c4b221bb
% https://www.oklink.com/amoy/tx/0x660e7ddf39f21279a1292de99e0a3128e5e0af28ce7269561142700ceea2cb5d
% total matic: 0.06276498

% create account:
% https://sepolia.etherscan.io/tx/0xaa549b9f913c2accb42471c01f4017684d150ee207394ff1fa259205166c89d2
% https://sepolia.etherscan.io/tx/0xc803f6d2f8c82030cf80f67b6841c249a2e79b79d9ce9d4de3f349f2e1caf52f
% total eth to create a account: 0.03412775571
% on destchain: 
% https://www.oklink.com/amoy/tx/0x130f8f1a674b61336e473a2081fb757970a3fccf09477bf2f23f93db6fbfe553
% total matic: 0.09301794048

% execute a transfer of eth (baseCHain):
% https://sepolia.etherscan.io/tx/0x235b0a8f5c58a3e34108101c5e150f9ad74cda92db21f525773fcabf71397870
% FIXME: must be checked. 0.00899325116681234 ETH

% execute a transfer of nft (baseCHain):

% bridge eth:


% bridge nft:

%%%%%%%%%
% NORMAL ETHEREUM
%%%%%%%%%
% Deployment of contracts: none
% create account: none
% execute a transfer of eth (baseCHain): 0.00265184
% execute a transfer of nft (baseCHain):
% bridge eth: none
% bridge nft: none

This section presents the outcomes of the implementation of Sapphire Wallet, a decentralized application aimed at facilitating seamless interaction with multiple blockchains.

% Improved UI/UX but still too slow to be used worldwide
The user interface and experience of Sapphire Wallet represent a significant improvement over traditional Ethereum Virtual Machine (EVM) networks. In particular, the user should not be aware of the underlying technology and does not need any technical knowledge to use the wallet. In fact, performing transactions on both \textit{BaseChain} and \textit{DestChain} are simple and intuitive. The \textit{recoverability} is largely improved, as the user can recover their account with the help of a friend, family member, or a trusted third party (\textit{Guardian}). 

However, despite these advancements, the current implementation faces challenges that limit its global scalability. Transaction speeds remain a concern, with operations such as NFT bridges between \textit{BaseChain} and \textit{DestChain} taking up to one minute, which may not meet the expectations for widespread adoption.

% Costs
In addition to low performance in terms of speed, the deployment costs associated with the Sapphire Wallet ecosystem are high. As shown in Table \ref{tab:costs}, the Sapphire Wallet ecosystem has deployment costs. In addition, for each created account there is the \textit{Proxy} wallet creation cost. In comparison, the actual EVM does not have these costs. 

The transactions costs ......

\begin{table}[H]
    \centering
    \begin{tblr}{
        width = \linewidth,
        colspec = {Q[420]Q[270]Q[270]},
        cell{1}{2} = {Gallery, halign = c, valign = m},
        cell{1}{3} = {Gallery, halign = c, valign = m},
        cell{2}{1} = {Gallery, halign = l, valign = m},
        cell{3}{1} = {Gallery, halign = l, valign = m},
        cell{4}{1} = {Gallery, halign = l, valign = m},
        cell{5}{1} = {Gallery, halign = l, valign = m},
        cell{6}{1} = {Gallery, halign = l, valign = m},
        cell{7}{1} = {Gallery, halign = l, valign = m},
        vline{2-4} = {1}{Silver},
        vline{-} = {2-7}{Silver},
        hline{1} = {2-3}{Silver},
        hline{2-8} = {-}{Silver},
        cell{2-7}{2,3} = {c, m},
    }
    & Sapphire Wallet ecosystem & Actual EVM \\
    Deploy infrastructure & A & B \\
    Deploy per user & C & D\\
    ETH transfer on Ethereum (BaseChain) & E & F\\
    NFT transfer on Ethereum (BaseChain) & G & H\\
    ETH bridge on Ethereum-Polygon (BaseChain-DestChain) & I & K\\
    NFT bridge on Ethereum-Polygon (BaseChain-DestChain) & L & M
    \end{tblr}
    \label{tab:costs}
    \caption{Costs comparison between Sapphire Wallet ecosystem and actual EVM}
    \end{table}


% "Separate the transaction system" like Argetn are not feasible in a long run, need to implement AA with protocol update

In general, Sapphire Wallet, as a \textit{Separate Transaction System}, has higher cost than the actual EVM. But, the features enabled by the Sapphire Wallet may be worth the cost. The main features are well described in the chapter \hyperref[sec:operational_flow_and_use_cases]{Operational Flow and Use Cases}. 

Looking ahead, the \textit{Separate Transaction System} approach is not feasible and a protocol update is needed to implement the full \textit{Account Abstraction} capabilities. In my opinion, the new \hyperref[subsubsec:erc-7702]{ERC-7702} standard is a good candidate to be implemented in the Ethereum network.

% USE CASE of this kind of system. "Private business that wants to pay for the gas of their users", granting security, privacy and scalability.

The main use case of \textit{Sapphire Wallet} may be for private businesses that wants to guarantee security, privacy and scalability to their users through Blockchain technology. The ability to manage transaction costs and simplify user interactions makes Sapphire Wallet a promising tool for businesses exploring blockchain solutions.
\chapter{Conclusions}
\label{sec:conclusions}



\bibliographystyle{unsrt}
\bibliography{bibliografia}
\end{document}
