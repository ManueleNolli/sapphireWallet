%% Esempio per lo stile supsi
\documentclass[twoside]{packages/supsistudent}

% per settare noindent
\setlength{\parindent}{0pt}

% Crea un capitolo senza numerazione che pero` appare nell'indice %
\newcommand{\problemchapter}[1]{%
  \chapter*{#1}%
  \addcontentsline{toc}{chapter}{#1}%
\markboth{#1}{#1}
}

% Numerazione delle appendici secondo norma
\addto\appendix{
\renewcommand{\thesection}{\Alph{chapter}.\arabic{section}}
\renewcommand{\thesubsection}{\thesection.\arabic{subsection}}}

\setcounter{secnumdepth}{5} 	%per avere più livelli nei titoli
\setcounter{tocdepth}{5}		%per avere più livelli nell'indice


\doctitle{Sapphire Wallet: Multi-Chain Account Abstraction}
\student{Nolli Manuele}
\supervisor{Gremlich Giuliano\\Guidi Roberto}
\cosupervisor{.}
\course{Master of Science in Engineering: Computer Science}
\anno{2023 - 2024}

% Image
\graphicspath{ {./images/} }

\addbibresource{linkography.lib}
\addbibresource{bibliography.lib}

\begin{document}

\pagenumbering{alph}
\maketitle
\onehalfspacing
\frontmatter


\pagenumbering{roman}
\tableofcontents
\listoffigures					% Opzionale
\listoftables					% Opzionale

\newpage
\mainmatter
\pagenumbering{arabic}
\setcounter{page}{1}

\problemchapter{Abstract}
\label{chap:abstract}

This paper presents the specialization project by Manuele Nolli, student of the Master of Science in Engineering (MSE) in Switzerland with a focus on Computer Science. The project, in collaboration with the Institute of Information Systems and Networking (ISIN) at the University of Applied Sciences and Arts of Southern Switzerland (SUPSI), aims to analyse the current state of Blockchain technology, specifically in the areas of Account Abstraction and Cross-Chain Communication. The core of the project is the development of a decentralized wallet, named Sapphire Wallet, which allows users to interact seamlessly with multiple blockchains without requiring technical knowledge.

The results demonstrate that Sapphire Wallet significantly improves user interface and experience compared to traditional Ethereum Virtual Machine (EVM) networks, offering intuitive transaction processes and enhanced account recoverability through trusted third parties (Guardians). However, the implementation faces challenges such as slow transaction speeds and high deployment costs, rendering it impractical for global adoption in its current form.

Despite these limitations, the Sapphire Wallet showcases the potential of the Account Abstraction paradigm and highlights the necessity for protocol updates to fully implement its capabilities. The paper also suggests that new blockchain systems, like ICP, which incorporate Account Abstraction and advanced Cross-Chain Communication paradigms, should be closely observed as they might represent the future of Blockchain technology.


\problemchapter{Introduction}
\label{chap:introduction}



\chapter{Motivation and Context}
\label{chap:motivation_and_context}
\chapter{Problem}
\label{chap:problem}

Blockchain technology presents significant barriers to mainstream adoption due to its complex \textit{user experience}. Interacting with a Blockchain typically requires users to navigate through technical processes such as signing transactions, paying fees, and waiting for confirmations. This complexity contrasts with the seamless user experience of traditional web 2.0 applications, where users can interact without any knowledge of the underlying technology.

The field is a very active research area, many solutions and chains have been proposed. In fact, millions of users and developers started to use them. \cite{blockchain-statistics} However, an important question arises when there are multiple chains: \textit{which one should I use?}

Interoperability is the answer to this question, enabling different blockchain networks to communicate effectively with each other. By achieving interoperability, the decision of which chain to use is not relevant and a business can benefit from the best features of each chain without losing market share. But how can this be achieved \textit{without experienced users?}

The objective of this Master's project is to analyse the current state of the art of Blockchain \textit{Interoperability} and \textit{Account Abstraction}, and to propose a solution that allows users to interact with multiple Blockchains eliminating the need for technical knowledge.

\chapter{State of the Art}
\label{chap:state_of_the_art}

This chapter presents the state of the art of Blockchain interoperability and account abstraction, with an emphasis on EVM-compatible Blockchains.\cite{evm} 

In the first section, the concept of Account Abstraction is introduced, with a focus on the evolution of the proposals, followed by an analysis of the Argent wallet, to understand how Account Abstraction can improve the user experience.

The second section presents the state of the art of Cross-Chain Communication, categorising the bridges according to the trust model, communication model, and asset transfer model. 

Finally, the Internet Computer Protocol (ICP) is presented, as it is a promising Blockchain solution for both Account Abstraction and Cross-Chain Communication. 


\section{Account Abstraction}
\label{sec:account_abstraction}

%What is an account?

The term \textit{Account Abstraction} refers to the ability to extend the capabilities of an account in a blockchain. In the context of Ethereum, there are two type of accounts: \cite{ethereum-accounts}

\begin{itemize}
    \item \textit{Externally Owned Accounts (EOA)}: controlled by anyone with the private key.
    \item \textit{Contract Accounts}: A deployed smart contract. Controlled by the code.
\end{itemize}

% Problem with EOA: fee, lost private key, etc.

EOAs are the only account type that can initiate transactions, while contract accounts can only \textit{react} \footnote{A smart contract can initiate a transaction by calling another smart contract, but the transaction is still initiated by an EOA.} to transactions, which makes it difficult to do batches of transactions and requires users to always keep an ETH balance to cover gas. Other limitations include the lack of recovery options, the need to pay a fee for each transaction, and the risk of losing the private key. Generalising, the user experience is not as seamless as in traditional web 2.0 applications. \cite{ethereum-account-abstraction}

% What is account abstraction? Use a contract accounts instead of EOAs.
% Why use account abstraction? different use cases: recovery, relaying transactions, etc.

Account abstraction is a way to solve these problems by allowing users to flexibly program more security and better user experiences into their accounts. This is achieved by \textit{abstracting} the account model, allowing users to use contract accounts instead of EOAs. The use of a new type of account named \textbf{Smart Contract Wallets} unlocks new possibilities, including: \cite{ethereum-account-abstraction}

\begin{itemize}
    \item \textit{Flexible security models}: 
    \item \textit{Recovery options}:
    \item \textit{Multi owner accounts}:
    \item \textit{Relayed transactions}:
    \item \textit{Batch/Multi transactions}:
    \item \textit{Innovative User Experience}:
\end{itemize}


\subsection{ERC standards}
\label{subsec:erc_standards}

EIP-86, EIP-2938, EIP-3078, EIP-4337, ERC-4337

\subsection{Argent}
\label{subsec:argent}

\section{Cross-Chain Communication}
\label{sec:cross_chain_communication}

\subsection{Bridge classification}
\label{subsec:bridge_classification}

\subsubsection{Trust model}
\label{subsubsec:trust_model}

\subsubsection{Communication model}
\label{subsubsec:communication_model}

\subsubsection{Asset transfer model}
\label{subsubsec:asset_transfer_model}


\section{IC: Internet Computer}
\label{sec:icp}

The Internet Computer (IC)\footnote{https://internetcomputer.org/} is a blockchain protocol developed by the DFINITY Foundation\footnote{https://dfinity.org/}, a Switzerland non-profit organization.  

IC's vision is that most of the world's software will be replaced by smart contracts. To realize that vision, IC is designed to make smart contracts as powerful as traditional software. \cite{icp-vision}

\subsection{Architecture}

The Internet Computer (IC) realizes the vision of a \textit{World Computer}, an open and secure blockchain-based network that can host programs and data in the form of smart contracts, perform computations on smart contracts in a secure and trustworthy way, and scale infinitely.\cite{icp-how-it-works}

Smart contracts on the Internet Computer are called \textbf{Canister} smart contracts, each consisting of a bundle of WebAssembly (Wasm) bytecode and storage. Each canister has its own, isolated, data storage that is only changed when the canister executes code. \cite{icp-how-it-works}

The IC is designed to be highly scalable and efficient in terms of hosting and executing canister smart contracts. The top-level building blocks of the IC are subnetworks, or \textbf{subnets}, the IC as a whole consists of many subnets, where each subnet is its own blockchain that operates concurrently and independently, but can communicate asynchronously with other subnets. Each subnet hosts canister smart contracts, up to a total of hundreds of gigabytes of replicated storage. A subnet consists of node machines, or nodes. Each node replicates all the canisters, state, and computation of the subnet using blockchain technology. This makes a subnet a blockchain-based replicated state machine, that is, a virtual machine that holds state in a secure, fault-tolerant, and non-tamperable manner. The computations of the canisters hosted on a subnet will proceed correctly and without stopping, even if some nodes on the subnet are faulty (either because they crash, or even worse, are hacked by a malicious party). New subnets can be created from nodes added to the IC to scale out the protocol, analogous to how traditional infrastructures such as public clouds scale out by adding machines. \cite{icp-architecture}

\textcolor{red}{Add figure}

\subsubsection{IC Protocol (ICP)}

\textcolor{red}{Go deeper or remove?}

The Internet Computer is created by the Internet Computer Protocol (ICP). The ICP is a 4-layer protocol that is running on the nodes of each subnet. The 4 layers are: \cite{icp-how-it-works}

\begin{itemize}
    \item \textit{Peer-to-peer}: Responsible for the secure and reliable communication between the nodes of a subnet. Using P2P, a node can broadcast a network message to all the nodes in the subnet.
    \item \textit{Consensus}: Allows the nodes to agree on the messages to be processed, as well as their ordering. Consensus is the component of the core IC protocol that drives the subnets of the IC.
    \item \textit{Message routing}: Receives a block of messages to be processed from consensus and places the messages into the input queues of the target canisters.
    \item \textit{Execution}: execute canister smart contract code.
\end{itemize}


\subsection{Internet Identity: Account Abstraction}

Internet Identity is the ICP's native form of identity. IC uses \textit{passkeys}, a unique public/private key pair stored in the secure hardware chip, to authenticate users. They offer a convenient and more secure alternative to passwords, enabling a password-free login experience for websites and applications. \cite{icp-identity}

Passkeys are based on the WebAuthentication (WebAuthn) cryptographic standard. When creating an account, the operating system generates a unique and app- or website-specific cryptographic key pair \textit{on the device}. \cite{icp-passkey}

To interact with a canister, users need to authenticate their actions using a cryptographic key pair (passkey). The network verifies the digital signature using the associated public key, ensuring that the action was indeed signed by the authorized user. If the verification is successful, the action is executed by the canister on the Internet Computer. \cite{icp-passkey}

Internet Identity offers several benefits over other blockchains account systems, some \hyperref[sec:account_abstraction]{\textit{Account Abstraction}} features are natively supported: \cite{icp-identity-technical}
\begin{itemize}
    \item \textit{Improved User Experience}: Authentication via cryptographic key on the device without the need of passwords. 
    \item \textit{Security}: Users can add many devices to the same identity \textit{Anchor}\footnote{An anchor is a number attached to a user's identity. This number is auto-generated by the system and there is only one per identity.}. A common example is that users may add their smartphone and desktop computer to an identity anchor. The authentication can be done using any of the devices.
    \item \textit{Recovery Mechanism}: There are two recovery mechanisms for recovery the \textit{Anchor}:
        \begin{itemize}
            \item \textit{Seed Phrase}: A user can select this option to generate a cryptographically-secure seed phrase that they can use to recover an Identity Anchor. 
            \item \textit{Security Key}: A dedicated security key can be used to recover an Identity Anchor in the event that a user loses access to their authorized devices. 
        \end{itemize}
\end{itemize}

\textcolor{red}{Who pays the fees/cicles? Canister?}

\subsection{Chain-key Cryptography: Interoperability}

\textit{Chain-key cryptography} enables subnets of the Internet Computer to jointly hold cryptographic keys, in a way that no small subset of potentially misbehaving nodes on the subnet can perform useful operations with the key, but the majority of honest nodes together can. \cite{icp-chain-key}

Moreover, with the combination of HTTPS outcalls feature and chain-key cryptography, the Internet Computer canisters can interact with external services and blockchains. \cite{icp-https-outcalls}

Indeed, one of the major benefits of chain-key cryptography is that it enables canister to \textit{natively create signed transactions on other blockchains} like Ethereum or Bitcoin \textbf{without a bridge}. \cite{icp-cross-chain-interoperability} 

For instance, ICP is integrated with the Bitcoin network using chain-key ECDSA signatures and a protocol-level integration, allowing for a canister to create a Bitcoin address, then send or receive bitcoin directly as if they were a regular Bitcoin user. ICP will also use chain-key ECDSA to facilitate an upcoming integration with Ethereum that will allow Ethereum smart contracts and digital assets like ERC-20 tokens to be used in ICP canisters. \cite{icp-cross-chain-interoperability}


\chapter{Problem Approach}
\label{chap:problem_approach}
\chapter{Results}
\label{chap:results}

%%%%%%%%%
% WITH SAPPHIRE
%%%%%%%%%

% Deployment of contracts: 
% https://sepolia.etherscan.io/tx/0xeca9946ce552bcdaf05bebc7e0708e0a83fb4b35a6c40e3cfbcf1298f5836fa0
% https://sepolia.etherscan.io/tx/0xfc5e5ad4553769e5e95985d568ee6417a9ed77222afda2b9f4aef74a25b28c06
% https://sepolia.etherscan.io/tx/0xfead6d10fe5190a3e6a85fa3d3c384ed6ae196db0b154ae0fb04c627931c3bab
% https://sepolia.etherscan.io/tx/0xa1a74bbafacbf28ac3d1d41d65ba5b8cbb9c6f2dc8ef9873a776b62b19a550fc
% https://sepolia.etherscan.io/tx/0x8cb2d7c617f3bb2d1e0cfe0b8fc2c87051e05e7ec060f876441de83c493260d0
% https://sepolia.etherscan.io/tx/0x2164279fbd3bd5c7305f269af0c5823447f4aa09d978f9b8a0252438bc0733e5
% https://sepolia.etherscan.io/tx/0xdc1455a9f9dcbd99c3a97fe8b1b4f9f2e9c7090347f58f3ab0f94301f619cb34
% https://sepolia.etherscan.io/tx/0xc34b7037aa27c4c0b775d6e09f1f489ef9904158f747eab723f013acf771f04d
% https://sepolia.etherscan.io/tx/0xf7801500c8e2761e4ffad8d84a06ba8fd2f611c0b832af7512968d53b6443193
% total eth to deploy: 0.12872771
% https://www.oklink.com/amoy/tx/0x43ae451a87a7ebd40c70f55554accf690bd7da81ebc35fe7ce234de7c4b221bb
% https://www.oklink.com/amoy/tx/0x660e7ddf39f21279a1292de99e0a3128e5e0af28ce7269561142700ceea2cb5d
% total matic: 0.06276498

% create account:
% https://sepolia.etherscan.io/tx/0xc75871d3be9c35b0b3412ae406d56559a18d957e3c9b8b9a11aa4bd44d61e115
% https://sepolia.etherscan.io/tx/0xae5026183af3a12f9ab6cebc7351290bfdd3769109ae29e048dc9e972af4a8d1
% total eth to create a account: 0.014
% on destchain: 
%https://www.oklink.com/amoy/tx/0x09c5d18e296a1ba860cf939a1b59de9924d64cb9140674aafe5a272907619313
% total matic: 0.018

% execute a transfer of eth (baseCHain):
% https://sepolia.etherscan.io/tx/0x8aa452cddc52dc9104b38272446688a2dc0f527d4271bda451a51fb9844758b5
% 0.003 ETH

% execute a transfer of nft (baseCHain):
% https://sepolia.etherscan.io/tx/0x438e6d330c46e1795380d82154fad342d1c06835baa81debca3dfb338df45a9a
%  0.0048 ETH

% bridge eth:
%https://sepolia.etherscan.io/tx/0x1a8235ad8deabb88326e476a5ece1aee8156ee1f2f9da4f43571cd7bfea09cc4
%https://www.oklink.com/amoy/tx/0x6261bdb038be42562fbdfb6e055ce3382163c356b55dc37c3df4c73ad1e7c310
%0.003 ETH + 0.001 MATIC  

% bridge nft:
%https://sepolia.etherscan.io/tx/0x086619799ec19fbb43bf47183279c28d0ffd5b95246389591511390970c36fcb
%https://www.oklink.com/amoy/tx/0x01110003842cd8764f9135e7ae62e1f1a97e7bf4d574553e4c7808472251d171
%0.0063 ETH + 0.0106 MATIC 

% send destchain matic:
%https://sepolia.etherscan.io/tx/0x3ce555b5f350647856d93bccd3bcbadc52cbe78d7446860e7629599f53268a3c
%https://www.oklink.com/amoy/tx/0xc67efbe8eee3ffe1392b3f17f75c09ef2be0a8c3b577ba97402ffb01f3e6fb55
%0.0086 ETH + 0.0017 MATIC 

% send destchain nft:
%https://sepolia.etherscan.io/tx/0x6c527aa87a2a0c66759a74a86a29dad431903365939307b3fbffe23ab5ae5023
%https://www.oklink.com/amoy/tx/0x87db8ecd117b6b96834368f5e5f6901f04bdc41376911b3e7c351a0bdb748530
%0.005 ETH + 0.0035 MATIC

% Add Guardian
%https://sepolia.etherscan.io/tx/0x35a71f6c720224dc524f75dc4a7a5915aa2012b05da59f7007c232b867d7ab46
% 0.0043 ETH 

% Remove Guardian
%https://sepolia.etherscan.io/tx/0x306901ca6bfcc2c5cd7318c1a01e81ff8a0f0683a1d3be6e4c9dd2403d0513e9
%0.0022 ETH

% Recovery
%https://sepolia.etherscan.io/tx/0x399bfd853c32563d3c2801ae93e1cf9fab51b0c6dc20dd538e7f6f7060e52da4
%0.003 ETH

This section presents the outcomes of the implementation of Sapphire Wallet, a decentralized application aimed at facilitating seamless interaction with multiple Blockchains.

% Improved UI/UX but still too slow to be used worldwide
The user interface and experience of Sapphire Wallet represent a significant improvement over traditional Ethereum Virtual Machine (EVM) networks. In particular, the user should not be aware of the underlying technology and does not need any technical knowledge to use the wallet. In fact, performing transactions on both \textit{BaseChain} and \textit{DestChain} are simple and intuitive. The \textit{recoverability} is largely improved, as the user can recover their account with the help of a friend, family member, or a trusted third party (\textit{Guardian}). 

However, despite these advancements, the current implementation faces challenges that limit its global scalability. Transaction speeds remain a concern, with operations such as NFT bridges between \textit{BaseChain} and \textit{DestChain} taking up to one minute, which may not meet the expectations for widespread adoption.

% Costs
In terms of costs, the Sapphire Wallet ecosystem requires an initial deployment of Smart Contracts and one Contract per user. Moreover, as shown in Table \ref{tab:costs}, Sapphire Wallet features increase the cost of transferring cryptocurrencies or NFTs. Due to its \textit{Separate Transaction System} nature, Sapphire Wallet has higher cost than the actual EVM. But, the features enabled by the Sapphire Wallet may be worth the cost. In fact, in the current state of EVM networks, features such as recoverability and interoperability are not achievable.

\begin{table}[H]
    \centering
    \begin{tblr}{
        width = \linewidth,
        colspec = {Q[420]Q[270]Q[270]},
        cell{1}{2} = {Gallery, halign = c, valign = m},
        cell{1}{3} = {Gallery, halign = c, valign = m},
        cell{2}{1} = {Gallery, halign = l, valign = m},
        cell{3}{1} = {Gallery, halign = l, valign = m},
        cell{4}{1} = {Gallery, halign = l, valign = m},
        cell{5}{1} = {Gallery, halign = l, valign = m},
        cell{6}{1} = {Gallery, halign = l, valign = m},
        cell{7}{1} = {Gallery, halign = l, valign = m},
        cell{8}{1} = {Gallery, halign = l, valign = m},
        cell{9}{1} = {Gallery, halign = l, valign = m},
        cell{10}{1} = {Gallery, halign = l, valign = m},
        cell{11}{1} = {Gallery, halign = l, valign = m},
        cell{12}{1} = {Gallery, halign = l, valign = m},
        vline{2-4} = {1}{Silver},
        vline{-} = {2-12}{Silver},
        hline{1} = {2-3}{Silver},
        hline{2-13} = {-}{Silver},
        cell{2-12}{2,3} = {c, m},
    }
    & Sapphire Wallet ecosystem & Actual EVM \\
    Deploy infrastructure & {0.129 ETH \\ 0.063 MATIC} & (1) \\
    Deploy per user & {0.014 ETH 0.018 MATIC} & (1)\\
    ETH transfer on Ethereum (BaseChain) & 0.003 ETH & 0.00085 ETH\\
    NFT transfer on Ethereum (BaseChain) & 0.0048 ETH & 0.00182 ETH \\
    ETH bridge on Ethereum-Polygon (BaseChain-DestChain) & {0.003 ETH \\ 0.0011 MATIC}& (2)\\
    NFT bridge on Ethereum-Polygon (BaseChain-DestChain) & {0.0063 ETH \\ 0.0106 MATIC} & (2)\\
    DestChain MATIC transfer & {0.0086 ETH \\ 0.0017 MATIC} & (2)\\
    DestChain NFT transfer & {0.005 ETH \\ 0.0035 MATIC} & (2)\\
    Add Guardian & 0.0043 ETH & (1)\\
    Remove Guardian & 0022 ETH & (1)\\
    Wallet recovery & 0.003 ETH & (1)\\
    \end{tblr}
    \label{tab:costs}
    \caption{Costs comparison between Sapphire Wallet ecosystem and actual EVM}

    (1) not necessary
    (2) not possible or realizable with external service

    \textit{BaseChain} is Sepolia (Ethereum) and \textit{DestChain} is Amoy (Polygon)
\end{table}



% "Separate the transaction system" like Argent are not feasible in a long run, need to implement AA with protocol update


Looking ahead, the \textit{Separate Transaction System} approach is not feasible and a protocol update is needed to implement the full \textit{Account Abstraction} capabilities. In my opinion, the new \hyperref[subsubsec:erc-7702]{ERC-7702} standard is a good candidate to be implemented in the Ethereum network.

% USE CASE of this kind of system. "Private business that wants to pay for the gas of their users", granting security, privacy and scalability.

For the moment, the main use case of \textit{Sapphire Wallet} may be for private businesses that wants to guarantee security, privacy and scalability to their users through Blockchain technology. The ability to manage transaction costs and simplify user interactions makes Sapphire Wallet a promising tool for businesses exploring blockchain solutions.

\chapter{Conclusions}
\label{sec:conclusions}

% Starting point to undesrtand the potential of the blockchain tecnology with AA
The Sapphire Wallet provides insight into what the future of blockchain technology could be. Indeed, it permits the interaction with multiple Blockchains in combination of a user experience never seen before. The \textit{Account Abstraction} paradigm is not a new concept in Ethereum, but the complexity of the implementation is high. In fact, until now, none of the proposals have been fully accepted by the Ethereum community. 
%ADD SOMETHING

The Sapphire Wallet is a proof of concept that shows the potential of the \textit{Account Abstraction} paradigm, but it is not feasible in the long run. The separation of the transaction system is not a viable solution, as it is too complex and expensive. The implementation of the \textit{Account Abstraction} paradigm requires a protocol update.

% New blockchain tecnology with different paradigms and with AA built it must be watched with attention like ICP

In the other hand, new Blockchain systems like \hyperref[sec:icp]{ICP} have \textit{Account Abstraction} and new \textit{Cross-chain communication} paradigms built-in. They must be watched with attention, as they could be the future of the Blockchain technology.


\printbibliography[title={Bibliography},nottype=online]
\printbibliography[title={Linkography},type=online]
\end{document}
